\documentclass{medkarta}
\addtext{На боль в поясничном отделе позвоночника, без иррадиации, потери чувствительности не отмечает, жалобы беспокоят около 6 часов, связывает с чрезмерными физическими нагрузками (разового характера), самостоятельно не лечился. Вызвал бригаду ОНМП для обезболивания.}
\addtext{В анамнезе:  остеохондроз позвоночника, люмбалгия- несколько раз в год. Операций и травм не было. Эпиданамнез не отягощен, за предела Москвы и РФ не выезжал более года, сезонные прививки отрицает. ЕМИАС, МГФОМС не информативен.}
\addtext{удовл.}
\addtext{ясное}
\addtext{15}
\addtext{вынужденное}
\addtext{ограничение в разгибании туловища из-за болевого синдрома}
\addtext{сухие}
\addtext{}
\addtext{обычной окраски}
\addtext{}
\addtext{}
\addtext{}
\addtext{}
\addtext{}
\addtext{ нет}
\addtext{ чистый}
\addtext{ не увеличены}
\addtext{ не увеличены}
\addtext{ нет}
\addtext{ нет}
\addtext{36.5}
\addtext{16}
\addtext{}
\addtext{нет}
\addtext{везикулярное}
\addtext{}
\addtext{}
\addtext{}
\addtext{}
\addtext{}
\addtext{По всем полям}
\addtext{}
\addtext{}
\addtext{}
\addtext{Нет}
\addtext{}
\addtext{}
\addtext{}
\addtext{Нет}
\addtext{}
\addtext{}
\addtext{Нет}
\addtext{легочный}
\addtext{}
\addtext{}
\addtext{}
\addtext{}
\addtext{по всем полям}
\addtext{}
\addtext{}
\addtext{}
\addtext{отсутствует}
\addtext{ нет}
\addtext{92}
\addtext{ритмичный}
\addtext{удовлетворительное}
\addtext{92}
\addtext{0}
\addtext{130/80}
\addtext{120/80}
\addtext{140/90}
\addtext{звучные}
\addtext{}
\addtext{}
\addtext{нет}
\addtext{ нет}
\addtext{нет}
\addtext{}
\addtext{нет}
\addtext{влажный}
\addtext{чистый}
\addtext{правильная}
\addtext{мягкий}
\addtext{не напряжен}
\addtext{безболезненный}
\addtext{во всех отделах}
\addtext{}
\addtext{ отрицательные}
\addtext{ выслушивается}
\addtext{ не увеличена}
\addtext{ не пальпируется}
\addtext{ нет}
\addtext{ регулярный,  оформлен }
\addtext{спокойное}
\addtext{}
\addtext{}
\addtext{ контактен}
\addtext{сохранена D = S}
\addtext{внятная}
\addtext{}
\addtext{OD = OS}
\addtext{обычные}
\addtext{ живая}
\addtext{ нет}
\addtext{ нет}
\addtext{}
\addtext{нет}
\addtext{нет}
\addtext{ выполняет правильно}
\addtext{ дизурии нет}
\addtext{ отрицательный с обеих сторон}
\addtext{визуально пояснично кресцовая область не изменена, умеренное напряжение мышц спины, паравертебральная пальпация болезненная, симптом Лассега отрицательный с обеих сторон. ВАШ 30\%}
\addtext{}
\addtext{ расспрос,{\par} Sol.Ketaroli 30mg/ml-1ml в/м в левую ягодичную область{\par}Оставлен на месте, рекомендовано обратиться в поликлинику{\par}{\par}через  мин: состояние удовлетворительно, АД 120/80 мм.рт.ст., ЧСС 88 в мин, пульс 88 в мин, ЧД 16 в мин, ВАШ 10\%}
\addtext{препараты НПВС 2-3 раза в день, консультация невролога в плановом порядке}
\addtext{ нет}
\addtext{6}
\addtext{1}
\addtext{1}
\addtext{1}
\addtext{1}
\addtext{1}
\addtext{0}
\addtext{1}
\addtext{0}
\addtext{0}
\addtext{0}
\addtext{0}
\addtext{0}
\addtext{0}
\addtext{1}
\addtext{0}
\addtext{}
\addtext{20.01.21 133678531}
\begin{document}
\relsize{-2}
\begin{zhaloby}
  \gettext{1}
\end{zhaloby}
\begin{anamnez}
  \gettext{2}
\end{anamnez}

\textbf{ОБЪЕКТИВНО:}
общее состояние (\underlinetext{\gettext{3}}{удовл.}, \underlinetext{\gettext{3}}{ср.тяжести}, \underlinetext{\gettext{3}}{тяжелое}, \underlinetext{\gettext{3}}{терминальное}). Сознание: \underlinetext{\gettext{4}}{ясное}, \underlinetext{\gettext{4}}{оглушенное}, \underlinetext{\gettext{4}}{сопор}, \underlinetext{\gettext{4}}{кома}. \gettext{5}

Положение \underlinetext{\gettext{6}}{активное}, \underlinetext{\gettext{6}}{пассивное}, \underlinetext{\gettext{6}}{вынужденное} \uline{\gettext{7}\hfill}

Кожные покровы: \underlinetext{\gettext{8}}{сухие}, \underlinetext{\gettext{9}}{влажные}, \underlinetext{\gettext{10}}{обычной окраски}, \underlinetext{\gettext{11}}{бледные}, \underlinetext{\gettext{12}}{гиперемия}, \underlinetext{\gettext{13}}{цианоз}, \underlinetext{\gettext{14}}{желтушность}\uline{ \gettext{15}\hfill}

Сыпь\uline{ \gettext{16}\hfill}
Зев\uline{ \gettext{17}\hfill}
Миндалины\uline{ \gettext{18}\hfill}

Лимфоузлы\uline{ \gettext{19}\hfill}
Пролежни\uline{ \gettext{20}\hfill}
Отеки\uline{ \gettext{21}\hfill}
t$^\circ$C\uline{ \gettext{22}\hspace{2cm}}

Органы дыхания: ЧДД \uline{ \gettext{23} }в мин., одышка \underlinetext{\gettext{24}}{эксператорная}, \underlinetext{\gettext{24}}{инспираторная}, \underlinetext{\gettext{24}}{смешанная}. Патологическое дыхание\uline{ \gettext{25}\hfill}

Аускультативно: \underlinetext{\gettext{26}}{везикулярное}, \underlinetext{\gettext{27}}{жесткое}, \underlinetext{\gettext{28}}{бронхиалоное}, \underlinetext{\gettext{29}}{пузрильное}, \underlinetext{\gettext{30}}{ослаблено}, \underlinetext{\gettext{31}}{отсутствует} в\uline{ \gettext{32}\hfill}

Хрипы \underlinetext{\gettext{33}}{сухие} (\underlinetext{\gettext{34}}{свистящие}, \underlinetext{\gettext{35}}{жужжащие}) в\uline{ \gettext{36}\hfill}

\underlinetext{\gettext{33}}{Влажные} (\underlinetext{\gettext{37}}{мелко-}, \underlinetext{\gettext{38}}{средне-}, \underlinetext{\gettext{39}}{крупнопузырчатые}) в\uline{ \gettext{40}\hfill}

\underlinetext{\gettext{41}}{Крепитация}, \underlinetext{\gettext{42}}{шум трения плевры} над\uline{ \gettext{43}\hfill}

Перкуторный звук \underlinetext{\gettext{44}}{легочный}, \underlinetext{\gettext{45}}{тимпанический}, \underlinetext{\gettext{46}}{коробочный}, \underlinetext{\gettext{47}}{притупленный}, \underlinetext{\gettext{48}}{тупой} над\uline{ \gettext{49}\hfill}

Кашель \underlinetext{\gettext{50}}{сухой}, \underlinetext{\gettext{51}}{влажный}, \underlinetext{\gettext{52}}{лающий}, \underlinetext{\gettext{53}}{отсутствует}. Мокрота\uline{ \gettext{54}\hfill}

\textbf{Органы кровообращения:} пульс\uline{ \gettext{55} }в мин., \underlinetext{\gettext{56}}{ритмичный}, \underlinetext{\gettext{56}}{аритмичный}, наполнение\uline{ \gettext{57} }ЧСС\uline{ \gettext{58} }в мин.

дефицит пульса\uline{ \gettext{59} }АД\uline{ \gettext{60} }привычное\uline{ \gettext{61} }максимальное\uline{ \gettext{62} }мм.рт.ст.

Тоны сердца \underlinetext{\gettext{63}}{звучные}, \underlinetext{\gettext{63}}{приглушены}, \underlinetext{\gettext{63}}{глухие}. Шум \underlinetext{\gettext{64}}{систолический}, \underlinetext{\gettext{65}}{диастолический} на\uline{ \gettext{66}\hfill}

проводится \uline{ \gettext{67}\hfill} \underlinetext{\gettext{68}}{Шум трения перикарда}. Акцент тона \uline{ \gettext{69}\hfill} на \uline{ \gettext{70}\hfill}

\textbf{Органы пищеварения:} Язык \underlinetext{\gettext{71}}{сухой}, \underlinetext{\gettext{71}}{влажный}, \underlinetext{\gettext{71}}{обложен} \uline{ \gettext{72}\hfill}

Живот форма\uline{ \gettext{73} }\underlinetext{\gettext{74}}{мягкий}, \underlinetext{\gettext{74}}{напряжен} в\uline{ \gettext{75}\hfill}

\underlinetext{\gettext{76}}{Безболезненный}, \underlinetext{\gettext{76}}{болезненный} в\uline{ \gettext{77}\hfill} Положительные симптомы (\underlinetext{\gettext{78}}{Образцова},

\underlinetext{\gettext{78}}{Ровзинга},\underlinetext{\gettext{78}}{Ситковского}, \underlinetext{\gettext{78}}{Ортнера}, \underlinetext{\gettext{78}}{Мерфи}, \underlinetext{\gettext{78}}{Мейо-Робсона}, \underlinetext{\gettext{78}}{Щеткина-Блюмберга}, \underlinetext{\gettext{78}}{Вааля})\uline{ \gettext{79}\hfill}

Перистальтика\uline{ \gettext{80}\hfill}Печень\uline{ \gettext{81}\hfill}Селезенка\uline{ \gettext{82}\hfill}

Рвота (частота)\uline{ \gettext{83}\hfill}Стул (консистенция, частота)\uline{ \gettext{84}\hfill}

\textbf{Нервная система:} Поведение \underlinetext{\gettext{85}}{спокойное}, \underlinetext{\gettext{86}}{беспокойное}, \underlinetext{\gettext{87}}{возбужден}. Контакт\uline{ \gettext{88}\hfill}

Чувствительность\uline{ \gettext{89}\hfill}Речь (\underlinetext{\gettext{90}}{внятная}, \underlinetext{\gettext{90}}{дизартрия}, \underlinetext{\gettext{90}}{афазия})\uline{ \gettext{91}\hfill}

Зрачки \gettext{92}, \underlinetext{\gettext{93}}{обычные}, \underlinetext{\gettext{93}}{широкие}, \underlinetext{\gettext{93}}{узкие}. Фотореакция\uline{ \gettext{94}\hfill}Нистагм\uline{ \gettext{95}\hfill}

Ассиметрия лица\uline{ \gettext{96}\hfill}Менингеальные симптомы (\underlinetext{\gettext{97}}{ригидность затылочных мышц},

\underlinetext{\gettext{97}}{Кенига}, \underlinetext{\gettext{97}}{Брудзинского})\uline{ \gettext{98}\hfill}Очаговые симптомы\uline{ \gettext{99}\hfill}

Координатные пробы\uline{ \gettext{100}\hfill}

Мочеполовая система\uline{ \gettext{101}\hfill}

Симптом поколачивания\uline{ \gettext{102}\hfill}

\begin{status}
  \gettext{103}
\end{status}
\begin{data}
  \gettext{104}
\end{data}
\begin{help}
  \gettext{105}
\end{help}
\begin{recomend}
  \gettext{106}
\end{recomend}
\begin{signal}
  \gettext{107}
\end{signal}

\textbf{Расходные материалы: Салфетки спиртовые №\useFRMfield{count1}[\gettext{108}],Бахилы\useFRMfield{count1}[\gettext{109}],Перчатки\useFRMfield{count1}[\gettext{110}],Маска\useFRMfield{count1}[\gettext{111}],Шпатель\useFRMfield{count1}[\gettext{112}],Чехол д.терм\useFRMfield{count1}[\gettext{113}],}

\textbf{Шприц 2,0 №\useFRMfield{count1}[\gettext{114}]5,0 №\useFRMfield{count1}[\gettext{115}]10,0 №\useFRMfield{count1}[\gettext{116}]20,0 №\useFRMfield{count1}[\gettext{117}], Катетер. куб.\useFRMfield{count1}[\gettext{118}]G Фикс. пластырь\useFRMfield{count1}[\gettext{119}]Скариф\useFRMfield{count1}[\gettext{120}]Тест полоски\useFRMfield{count1}[\gettext{121}],Пакет мед.отхлд.\useFRMfield{count1}[\gettext{122}]Маска}

\textbf{для небулайзера}\useFRMfield{count1}[\gettext{123}]

\uline{\gettext{124} \hfill}

\textbf{Дата и номер наряда\uline{ \gettext{125} \hfill}Подпись\uline{ \hfill}Карту проверил\uline{ \hfill}}

\end{document}
