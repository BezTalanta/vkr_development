\documentclass{medkarta}
\addtext{жалоб не предъявляет из-за тяжести состояния.}
\addtext{Со слов жены и сиделки в течении 2 дней ухудшилось состояние в виде нарастающей одышки в покое, общей слабости, неоднократно вызывали СМП, однако больной от госпитализации отказался, с 08.02.21 г. постепенно перестал отвечать на вопросы. Из мед. документов: выписка из истории болезни от 04.05.16г. : ИБС. Диффузный мелкоочаговый кардиосклероз. Синдром слабости синусового узла: тахи-бради вариант. Синусовая брадикардия. Преходящая СА-блокада 2 ст, Мобиц 2 с эпизодами синус ареста и паузами асистолии до 3 сек. и синдромом Морганьи-Адамса-Стокса.Пароксизмальная форма фибрилляции предсердий, тахисистолический вариант. Частая предсердная экстрасистолия. Операция- имплантация двухкамерной ЭКС от 04.05.16г.в режиме стимулляции DDD-60 имп/мин. ХСН 2А стадии. ФК-2. Гипертоническая болезнь 2ст, со слов длительно страдал варикозной болезнью нижних конечностей, трофические язвы. Является маломобильным пациентом в течении 5 лет, инв 1 гр. Травмы родственники отрицают. Эпиданамнез не отягощен. ЕМИАС, МГФОМС доступен}
\addtext{терминальное}
\addtext{кома}
\addtext{3 балла по ШКГ}
\addtext{пассивное}
\addtext{на спине}
\addtext{сухие}
\addtext{}
\addtext{}
\addtext{бледные}
\addtext{}
\addtext{цианоз}
\addtext{}
\addtext{цианоз носогубного треугольника, ногтевых фаланг}
\addtext{ нет}
\addtext{не осмотрен по тяжести состояния}
\addtext{не осмотрен по тяжести состояния}
\addtext{ не увеличены}
\addtext{ нет}
\addtext{голени, стопы}
\addtext{36.1}
\addtext{24}
\addtext{смешанная}
\addtext{нет}
\addtext{}
\addtext{жесткое}
\addtext{}
\addtext{}
\addtext{}
\addtext{}
\addtext{по всем полям}
\addtext{Влажные}
\addtext{}
\addtext{}
\addtext{нет}
\addtext{мелко-}
\addtext{средне-}
\addtext{крупнопузырчатые}
\addtext{до середины лопаток}
\addtext{}
\addtext{}
\addtext{нет}
\addtext{}
\addtext{}
\addtext{}
\addtext{притупленный}
\addtext{}
\addtext{в нижних отделах с обеих сторон}
\addtext{}
\addtext{}
\addtext{}
\addtext{отсутствует}
\addtext{ нет}
\addtext{65}
\addtext{ритмичный}
\addtext{слабое}
\addtext{65}
\addtext{0}
\addtext{110/60}
\addtext{130/80}
\addtext{190/100}
\addtext{глухие}
\addtext{систолический}
\addtext{}
\addtext{верхушке}
\addtext{ нет}
\addtext{нет}
\addtext{I}
\addtext{аорте}
\addtext{влажный}
\addtext{обложен беловатым налетом}
\addtext{правильная}
\addtext{мягкий}
\addtext{во всех отделах}
\addtext{безболезненный}
\addtext{больной на пальпацию не реагирует}
\addtext{}
\addtext{на пальпацию не реагирует}
\addtext{ выслушивается}
\addtext{+3 см от края реберной дуги, на пальпацию не реагирует }
\addtext{ не пальпируется}
\addtext{ нет}
\addtext{на момент осмотра нет}
\addtext{спокойное}
\addtext{}
\addtext{}
\addtext{контакта нет из-за тяжести состояния}
\addtext{сохранена D = S}
\addtext{внятная}
\addtext{нет}
\addtext{OD = OS}
\addtext{обычные}
\addtext{ живая}
\addtext{ нет}
\addtext{не выявлена}
\addtext{}
\addtext{не выявлены}
\addtext{не выявлены}
\addtext{не проводились из-за тяжести состояния}
\addtext{ мочится в памперс, моча темно желтого цвета}
\addtext{не проводился из-за тяжести состояния}
\addtext{Вес=72кг, Рост=на вид 170см, Костно-мышечная система без повреждений, на стопах повязки, под ними трофические язвы. ШКГ 3 балла на основании: отсутствие открывания глаз, отсутствие речевого ответа, отсутствие двигательного ответа. ШОКС 10 баллов, на основании: одышка в покое, находится на функциональной еровати с приподнятым головным концом, хрипы до середины лопаток, увеличение печени до 5 см, отеки конечностей.ФК111ст.}
\addtext{SpO2 88\%;  ЭКГ (08.02.21г. 13:17) ритм обусловлен ЭКС с ЧЖС 65 в мин,  ЭОС- отклонена вправо, сравнению с предыдущими ЭКГ ( бр. СМП) без отрицательной динамики.}
\addtext{С первой попытки, без технических сложностей установлен в/в катетер G 18 в v.Cubitali medii  справа. Sol. Natrii chloridi 0,9\% - 250 ml, со скоростью 40 кап в мин. Sol. Furosemidi 20 mg/ml- 4ml + Sol. Natrii chloridi 0,9\%- 10ml в/ в болюсно. Больной пришел с сознание на несколько минут. Проведен дистанционный консилиум с Зав ОНМПВиДН 9 Саидходжаевым С.С., родственникам пациента доведено о тяжелом состоянии пациента, настоятельно рекомендовано госпитализироваться в реанимационное отделение, однако родственники - законные представители, категорически отказались, мотивируя желанием самого пациента  "умереть дома". В ходе уговоров о мед. эвакуации в 13:30 перестал дышать, пульс на сонных артериях не определяется, на мониторе ЭКГ: ритм ЭКС. Констатирована клиническая смерть. Пациент уложен на спину, незамедлительно, в 13:30 начаты реанимационные мероприятия, ( см. протокол СРЛ). Вызов бригады СМП ( на себя).{\par}{\par}Оставлен на месте, рекомендовано обратиться в поликлинику{\par}{\par}14:00 пациент передан бригаде СМП 2-281, в состоянии клинической смерти. На ЭКГ в стандартных отведениях: ритм обусловлен ЭКС}
\addtext{}
\addtext{ нет}
\addtext{10}
\addtext{1}
\addtext{1}
\addtext{1}
\addtext{1}
\addtext{1}
\addtext{}
\addtext{1}
\addtext{1}
\addtext{}
\addtext{1}
\addtext{1}
\addtext{}
\addtext{}
\addtext{1}
\addtext{}
\addtext{воздуховод-1}
\addtext{20.02.21 ЛЖ нед, КОМА, КЛИН. СМЕРТЬ,РЕАН}
\begin{document}
\relsize{-2}
\begin{zhaloby}
  \gettext{1}
\end{zhaloby}
\begin{anamnez}
  \gettext{2}
\end{anamnez}

\textbf{ОБЪЕКТИВНО:}
общее состояние (\underlinetext{\gettext{3}}{удовл.}, \underlinetext{\gettext{3}}{ср.тяжести}, \underlinetext{\gettext{3}}{тяжелое}, \underlinetext{\gettext{3}}{терминальное}). Сознание: \underlinetext{\gettext{4}}{ясное}, \underlinetext{\gettext{4}}{оглушенное}, \underlinetext{\gettext{4}}{сопор}, \underlinetext{\gettext{4}}{кома}. \gettext{5}

Положение \underlinetext{\gettext{6}}{активное}, \underlinetext{\gettext{6}}{пассивное}, \underlinetext{\gettext{6}}{вынужденное} \uline{\gettext{7}\hfill}

Кожные покровы: \underlinetext{\gettext{8}}{сухие}, \underlinetext{\gettext{9}}{влажные}, \underlinetext{\gettext{10}}{обычной окраски}, \underlinetext{\gettext{11}}{бледные}, \underlinetext{\gettext{12}}{гиперемия}, \underlinetext{\gettext{13}}{цианоз}, \underlinetext{\gettext{14}}{желтушность}\uline{ \gettext{15}\hfill}

Сыпь\uline{ \gettext{16}\hfill}
Зев\uline{ \gettext{17}\hfill}
Миндалины\uline{ \gettext{18}\hfill}

Лимфоузлы\uline{ \gettext{19}\hfill}
Пролежни\uline{ \gettext{20}\hfill}
Отеки\uline{ \gettext{21}\hfill}
t$^\circ$C\uline{ \gettext{22}\hspace{2cm}}

Органы дыхания: ЧДД \uline{ \gettext{23} }в мин., одышка \underlinetext{\gettext{24}}{эксператорная}, \underlinetext{\gettext{24}}{инспираторная}, \underlinetext{\gettext{24}}{смешанная}. Патологическое дыхание\uline{ \gettext{25}\hfill}

Аускультативно: \underlinetext{\gettext{26}}{везикулярное}, \underlinetext{\gettext{27}}{жесткое}, \underlinetext{\gettext{28}}{бронхиалоное}, \underlinetext{\gettext{29}}{пузрильное}, \underlinetext{\gettext{30}}{ослаблено}, \underlinetext{\gettext{31}}{отсутствует} в\uline{ \gettext{32}\hfill}

Хрипы \underlinetext{\gettext{33}}{сухие} (\underlinetext{\gettext{34}}{свистящие}, \underlinetext{\gettext{35}}{жужжащие}) в\uline{ \gettext{36}\hfill}

\underlinetext{\gettext{33}}{Влажные} (\underlinetext{\gettext{37}}{мелко-}, \underlinetext{\gettext{38}}{средне-}, \underlinetext{\gettext{39}}{крупнопузырчатые}) в\uline{ \gettext{40}\hfill}

\underlinetext{\gettext{41}}{Крепитация}, \underlinetext{\gettext{42}}{шум трения плевры} над\uline{ \gettext{43}\hfill}

Перкуторный звук \underlinetext{\gettext{44}}{легочный}, \underlinetext{\gettext{45}}{тимпанический}, \underlinetext{\gettext{46}}{коробочный}, \underlinetext{\gettext{47}}{притупленный}, \underlinetext{\gettext{48}}{тупой} над\uline{ \gettext{49}\hfill}

Кашель \underlinetext{\gettext{50}}{сухой}, \underlinetext{\gettext{51}}{влажный}, \underlinetext{\gettext{52}}{лающий}, \underlinetext{\gettext{53}}{отсутствует}. Мокрота\uline{ \gettext{54}\hfill}

\textbf{Органы кровообращения:} пульс\uline{ \gettext{55} }в мин., \underlinetext{\gettext{56}}{ритмичный}, \underlinetext{\gettext{56}}{аритмичный}, наполнение\uline{ \gettext{57} }ЧСС\uline{ \gettext{58} }в мин.

дефицит пульса\uline{ \gettext{59} }АД\uline{ \gettext{60} }привычное\uline{ \gettext{61} }максимальное\uline{ \gettext{62} }мм.рт.ст.

Тоны сердца \underlinetext{\gettext{63}}{звучные}, \underlinetext{\gettext{63}}{приглушены}, \underlinetext{\gettext{63}}{глухие}. Шум \underlinetext{\gettext{64}}{систолический}, \underlinetext{\gettext{65}}{диастолический} на\uline{ \gettext{66}\hfill}

проводится \uline{ \gettext{67}\hfill} \underlinetext{\gettext{68}}{Шум трения перикарда}. Акцент тона \uline{ \gettext{69}\hfill} на \uline{ \gettext{70}\hfill}

\textbf{Органы пищеварения:} Язык \underlinetext{\gettext{71}}{сухой}, \underlinetext{\gettext{71}}{влажный}, \underlinetext{\gettext{71}}{обложен} \uline{ \gettext{72}\hfill}

Живот форма\uline{ \gettext{73} }\underlinetext{\gettext{74}}{мягкий}, \underlinetext{\gettext{74}}{напряжен} в\uline{ \gettext{75}\hfill}

\underlinetext{\gettext{76}}{Безболезненный}, \underlinetext{\gettext{76}}{болезненный} в\uline{ \gettext{77}\hfill} Положительные симптомы (\underlinetext{\gettext{78}}{Образцова},

\underlinetext{\gettext{78}}{Ровзинга},\underlinetext{\gettext{78}}{Ситковского}, \underlinetext{\gettext{78}}{Ортнера}, \underlinetext{\gettext{78}}{Мерфи}, \underlinetext{\gettext{78}}{Мейо-Робсона}, \underlinetext{\gettext{78}}{Щеткина-Блюмберга}, \underlinetext{\gettext{78}}{Вааля})\uline{ \gettext{79}\hfill}

Перистальтика\uline{ \gettext{80}\hfill}Печень\uline{ \gettext{81}\hfill}Селезенка\uline{ \gettext{82}\hfill}

Рвота (частота)\uline{ \gettext{83}\hfill}Стул (консистенция, частота)\uline{ \gettext{84}\hfill}

\textbf{Нервная система:} Поведение \underlinetext{\gettext{85}}{спокойное}, \underlinetext{\gettext{86}}{беспокойное}, \underlinetext{\gettext{87}}{возбужден}. Контакт\uline{ \gettext{88}\hfill}

Чувствительность\uline{ \gettext{89}\hfill}Речь (\underlinetext{\gettext{90}}{внятная}, \underlinetext{\gettext{90}}{дизартрия}, \underlinetext{\gettext{90}}{афазия})\uline{ \gettext{91}\hfill}

Зрачки \gettext{92}, \underlinetext{\gettext{93}}{обычные}, \underlinetext{\gettext{93}}{широкие}, \underlinetext{\gettext{93}}{узкие}. Фотореакция\uline{ \gettext{94}\hfill}Нистагм\uline{ \gettext{95}\hfill}

Ассиметрия лица\uline{ \gettext{96}\hfill}Менингеальные симптомы (\underlinetext{\gettext{97}}{ригидность затылочных мышц},

\underlinetext{\gettext{97}}{Кенига}, \underlinetext{\gettext{97}}{Брудзинского})\uline{ \gettext{98}\hfill}Очаговые симптомы\uline{ \gettext{99}\hfill}

Координатные пробы\uline{ \gettext{100}\hfill}

Мочеполовая система\uline{ \gettext{101}\hfill}

Симптом поколачивания\uline{ \gettext{102}\hfill}

\begin{status}
  \gettext{103}
\end{status}
\begin{data}
  \gettext{104}
\end{data}
\begin{help}
  \gettext{105}
\end{help}
\begin{recomend}
  \gettext{106}
\end{recomend}
\begin{signal}
  \gettext{107}
\end{signal}

\textbf{Расходные материалы: Салфетки спиртовые №\useFRMfield{count1}[\gettext{108}],Бахилы\useFRMfield{count1}[\gettext{109}],Перчатки\useFRMfield{count1}[\gettext{110}],Маска\useFRMfield{count1}[\gettext{111}],Шпатель\useFRMfield{count1}[\gettext{112}],Чехол д.терм\useFRMfield{count1}[\gettext{113}],}

\textbf{Шприц 2,0 №\useFRMfield{count1}[\gettext{114}]5,0 №\useFRMfield{count1}[\gettext{115}]10,0 №\useFRMfield{count1}[\gettext{116}]20,0 №\useFRMfield{count1}[\gettext{117}], Катетер. куб.\useFRMfield{count1}[\gettext{118}]G Фикс. пластырь\useFRMfield{count1}[\gettext{119}]Скариф\useFRMfield{count1}[\gettext{120}]Тест полоски\useFRMfield{count1}[\gettext{121}],Пакет мед.отхлд.\useFRMfield{count1}[\gettext{122}]Маска}

\textbf{для небулайзера}\useFRMfield{count1}[\gettext{123}]

\uline{\gettext{124} \hfill}

\textbf{Дата и номер наряда\uline{ \gettext{125} \hfill}Подпись\uline{ \hfill}Карту проверил\uline{ \hfill}}

\end{document}
